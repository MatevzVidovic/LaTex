\documentclass[12pt, letterpaper, twoside]{article}    %report   ,  a4paper
\usepackage[utf8]{inputenc}
\usepackage{mathtools}
\usepackage{amsmath}

\title{First document}
\author{Hubert Farnsworth\thanks{funded by the Overleaf team}}
\date{February 2017}


% clicking f1 does quickbuild, which is the whole compilation at once




\def\A{
\begin{bmatrix}
    x_1 & x_2 & \cdots & x_N
\end{bmatrix}}

\def\B{
\begin{bmatrix}
    ax_0 + bx_1 \\
    ax_1 + bx_2 \\
    \vdots \\
    x_{N-1} + x_N
\end{bmatrix}}

\def\C{
\begin{bmatrix}
    z_1 \\
    z_2 \\
    \vdots \\
    z_N
\end{bmatrix}}


\begin{document}

\maketitle


Using linebreaks:\\
This is how you break a line without starting a new paragraph.

Empty lines between two blocks of text make the second block into a new paragraph.

\hfill \break
And this is how blank space is inserted.


\hfill \break

And this is how blank space is inserted while starting a new paragraph.




\hfill \break

Packages mathtools and amsmath are needed for matricies and such.


\hfill \break

The dollar sign is used for inline math equations.
The escape character \textbackslash followed by [ starts it's own block of math and it is ended by \textbackslash]    This way math is written.




Subscripts in math mode are written as $a_b$ and superscripts are written as $a^b$. These can be combined an nested to write expressions such as

\[ T^{i_1 i_2 \dots i_p}_{j_1 j_2 \dots j_q} = T(x^{i_1},\dots,x^{i_p},e_{j_1},\dots,e_{j_q}) \]
 
We write integrals using $\int$ and fractions using $\frac{a}{b}$. Limits are placed on integrals using superscripts and subscripts:

\[ \int_0^1 \frac{dx}{e^x} =  \frac{e-1}{e} \]

Lower case Greek letters are written as $\omega$ $\delta$ etc. while upper case Greek letters are written as $\Omega$ $\Delta$.

Mathematical operators are prefixed with a backslash as $\sin(\beta)$, $\cos(\alpha)$, $\log(x)$ etc.


\hfill \break


Splitting a line:
\begin{equation}
\begin{split}
F = \{F_{x} \in  F_{c} &: (|S| > |C|) \\
 &\quad \cap (\text{minPixels}  < |S| < \text{maxPixels}) \\
 &\quad \cap (|S_{\text{conected}}| > |S| - \epsilon) \}
\end{split}
\end{equation}




Vectors and matices:

$\vec{AB} = 0_E$ \\
$\overrightarrow{AB} = 0_E$
$$ \left| \vec{a} \right| $$
$$ \vec{p}\times \vec{q}=|\vec{p}|\vec{q}|sin\theta \hat{n} $$
$$ \vec{p}\cdot\vec{q}=|\vec{p}|\vec{q}|cos\theta $$

And vectors are simply written as one-column matices.

$\begin{bmatrix}
    x_{1} \\
    x_{2} \\
    \vdots \\
	x_{m}
\end{bmatrix}$



% https://www.overleaf.com/learn/latex/Matrices
\[
\begin{matrix}
1 & 2 & 3\\
a & b & c
\end{matrix}
\]

\[
\begin{pmatrix}
1 & 2 & 3\\
a & b & c
\end{pmatrix}
\]

$\begin{bmatrix}
1 & 2 & 3\\
a & b & c
\end{bmatrix}$

$\begin{vmatrix}
1 & 2 & 3\\
a & b & c
\end{vmatrix}$

$\begin{Vmatrix}
1 & 2 & 3\\
a & b & c
\end{Vmatrix}$








\begin{equation}
y =\A \left( \B - \C \right)
\end{equation}








Here i have to use "aligned". On the internet they say "align" is used.
I don't know, but the error said to use aligned and it works now.

\[
  \begin{aligned}
    y &= (x_{1},x_{2},\cdots, x_{N})
        \begin{pmatrix}
          \begin{bmatrix}
           ax_{0} + bx_{1} \\           
           \vdots \\
           ax_{n-1}+bx_{n}
          \end{bmatrix} -
          \begin{bmatrix}
           z_{1} \\
           \vdots \\
           z_{n}
         \end{bmatrix}
    \end{pmatrix}
  \end{aligned}
\]







\end{document}